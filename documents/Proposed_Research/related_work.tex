\section{Related Work}

There has been notable previous work on static analysis for side channel detection. One type of approach is that of by Pasareanu et al. \cite{smtmulti} and Bang et al. \cite{BAQ16}. They both use a combination of symbolic execution and model counting to determine the probability of a particular execution path and tools from information theory to determine the amount of leakage based on those probabilities. This kind of approach suffers due to its reliance on symbolic execution which does not scale to large applications. Our approach could be considered complimentary to theirs. Using PCA, we can reduce a complicated program to a potentially much smaller set of suspicious methods for which symbolic execution might be scalable. Work by Zhang et al. on Sidebuster \cite{sidebuster} details a static approach to detecting side channels in web applications. Their work involves taint analysis to determine whether a secret value has propagated to a conditional branching statement. If it has, the number of remote procedure calls in each branch are compared. If the number of calls differs, the site is marked as a potential side channel. Both this approach and ours are built on detecting differences in branches resultant from conditionals dependent on the secret value, but the symbolic cost expressions key to PCA make it a much more expressive method. Another static approach is presented by Antopolous et al. in their tool Blazer \cite{decomposition}. Their approach involves recursively partitioning the set of possible traces through the program such that all traces in a given partition take approximately the same amount of time. Other types of static analysis techniques are particular to specific types of side channels, such as CacheAudit \cite{cacheaudit}, or are type-based analyses \cite{fixbyconstant,hedin}.

%Previous work on side channel detection and quantification can be broken down into static and dynamic approaches. Dynamic approaches, such as presented by Chothia et al. in their tool LeakWatch \cite{leakwatch} estimate the amount on leakage in a given program through statistical analysis \cite{formalbounds, stats} done on the repeated trials of the program using independently drawn input samples. \todo{probably remove all dynamic stuff} Our work is more closely related to other static techqniues to detecting side channels. 