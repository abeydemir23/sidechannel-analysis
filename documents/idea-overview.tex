\documentclass{article} 

\begin{document}     
\title{Side-channels in Runtime Systems:\\ An Idea Overview}
\author{Tegan Brennan\qquad Miroslav Gavrilov}
\maketitle

\section{Side-channel Overview}

A practically useful system always presents a complicated landscape, in an
information-theoretical view: to be deemed ``useful", computations have certain
characteristic behaviours they should follow (e.g. be of a certain complexity
class, execute at a certain speed, etc.), which more or less exerts certain
properties of the physical implementation of the system in question. By not
focusing on the result of the computation, but rather on the side-effects
manifest in this physical implementation, one can gain knowledge of some
implementational details, and thus the structure of the computation itself. This
is referred to as using a side-channel to learn things. Similarly, we say that
looking at the computation directly is a main-channel.

Side-channel exploration and exploitation have mostly been a part of the realm
of cryptography, but with the coming of Web 2.0 and mobile technologies, as well
as the exponential increase in processing power compared to power output or
computation time required, the possibility of finding a side-channel grew
significantly. One of the foci of side-channel analyses is preventing side-
channel attacks, in which an attacker is using one or potentially more side-
channels to discover some private data or gain insight into a hidden process.
That being as important as it is, side-stepping security, one (friendly
individual) can learn a lot about a system by observing side-channels. For
example,  timing side-channels present a view into different execution times of
a program, most frequently dependent only on the input passed into it. If the
program isn't observably polynomial with the size of the inputs (which would be
a trivial side-channel to catch), but we do see a pattern occuring with
differing inputs, we can conclude that there exists a branch in the program
structure which uses the input as a choice point: a certain class of inputs
activates one path in that branch, leading to some execution time $t_1$, while
the other leads to an observably different execution time $t_2$. This, for
example, reflects a very common coding practice that is used widely in
imperative programming, namely: early returns. In many environemnts where early
returns are predominant,  programmers value optimizing for speed, but such
optimizations could never cover the space of all programs for which they are
implemented, for all inputs\footnote{If an optimization could ever cover the
space of all programs for which it is implemented, for all inputs, that means
that it  would collapse the whole complexity class in which it is by one and
lead to the side-channel  itself not being recognizable.}, making it possible
for analyses to get knowledge about the system under test without breaking the
seal on the black-box.

This phenomenon builds a kind of philosophical construct similar to the Heisenberg
uncertainty principle wherein the limitations of our physical reality are stopping
us from being both optimized (take less time in most cases) and, for example, 
completely computationally private (as observed in \cite{archlab-timing-14}). The
case in which the least side-channel activity is present is the case in which the 
lengths of all the possible routes of execution of the main-channel are of similar
size, and thus observation becomes noisy in the presence of other physical factors.
This, unsurprisingly, is the case in which there is no optimization made and all the
lengths are similar to the longest one.

\subsection{Analysing runtime systems}

Runtime systems are complex architectures built for speed, and as such are a
logical source of many observable side-channels. Given that they operate over
different but generally equally-powerful languages, our main field of
exploration would cover the following questions: 
\begin{itemize}     
	\item{Given two programs written in two languages, recognized by some 
	dependable method of semantic clone detection\cite{sem-clo-det} as semantic 
	clones, how many of the side-channels arising from their execution are 
	recognizable as cloned channels?}
	\item{Given two runtimes running the same language, and a program in that 
	language, how many of the side-channels in that program's execution are
	cloned channels across runtimes?} 
	\item{Given one runtime with several languages executing in it, and a program's 
	semantic clones in those languages, how many of the observable side-channels 
	are also cloned channels across that runtime?}
	\item{Given a runtime's different levels of optimization and JIT/GC settings,
	for an observable side-channel, how does tweaking those setting change its
	strength?}
\end{itemize}
The answers to these questions could give an idea of how certain implementation
details decide on how suitable a runtime or language is for tasks that need a level
of discretion in lieu of a potential side-channel, as well as how they could 
decide some further optimizations.

\pagebreak
\begin{thebibliography}{9}
\bibitem{archlab-timing-14}
Jason Oberg, Sarah Meiklejohn, Timothy Sherwood, and Ryan Kastner.
\textit{Leveraging Gate-Level Properties to Identify Hardware Timing Channels}
http://ieeexplore.ieee.org/document/6879637/?reload=true

\bibitem{sem-clo-det}
Mark Gabel, Lingxiao Jiang, and Zhendong Su.
\textit{Scalable detection of semantic clones}
http://dl.acm.org/citation.cfm?id=1368132
\end{thebibliography}
\end{document}
