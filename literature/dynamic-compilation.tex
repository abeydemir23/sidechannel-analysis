
\section{D. Page -- A Note on Side-Channels Resulting \\ from Dynamic Compilation}

\subsection{Main Goals}

The main goal of this paper is to determine whether the dynamic compilation of Java programs can
introduce side channel vulnerabilities into the executed code when there were
none in the original code. 

In attempting to extract higher levels of perfomancing, a dynamic compiler
might transform a secure program into an insecure one. This is particularly
concerning as the programmer might have no idea that such a vulnerability was
introduced. 

\subsection{Case Study -- Elliptic Curve Cryptography}

The double-and-add method is an additive version of binary exponentiation used
in ECC based systems. Generally, the profiles of the addition and doubling
phases are distinguishable from one another. An attacker can thus determine the
sequence of additions and doubling, allowing them to recover the
secret key used in encryption. One way to eliminate this side channel is to
always perform both an addition and doubling, but this is very inefficient.
Another, more commonly used approach, is to split the more expensive addition
step into two parts, each of which resembles the doubling step in terms of
execution profile. 

A java implementation of this side channel resistant version of the
double-and-add program was analyzed in order to determine the effect of
optimization on side channel vulnerabilities. It was run on the Jikes Research
Virtual Machine (RVM) with the default options for the adaptive optimization
system.
Because the doubling method was called more frequently than the addition
methods, it was identified as a hotspot in the program and was optimized by the
RVM. The addition method was eventualy optimized as well, but nevertheless
there is a window in which the optimized double method is in use at the same
time was the unoptimized addition method. The profiles of these two methods are
very distinguishable. Even once the addition method is optimized, the two are
distinguishable, though less so, due to differing results of optimization.
The RVM adaptive optimization system therefore introduced a side channel
vulnerability when there was none in the original code. 

\subsubsection{Possible Solutions}
One possible patch highlighted in the paper is to allow code fragments to be
annotated with information that allows the virtual machine to avoid introducing
vulnerabilities into the code. A basic solution would be to include an
annotation that instructs the dynamic compiler to leave the associated method
alone. This of course trades performance for security. Another approach would
be to specify that dynamic compilation is permitted provided that the two
methods still match each other in terms of cost. While more work is needed to
determine how such a compilation phase might occur, there is already
pre-existing work that allows for the detection of any mismatch between the
results of re-compilation that might be reusable for this task. 
 
\subsection{Conclusion}
Optimizations introduced by dynamic compilation can indeed introduce side
channels into code when there was none in the souce. 
