
\section{S. Crane, A. Homescu, et al -- Thwarting Cache Side-Channel Attacks Through Dynamic Software Diversity}

\subsection{Main Goals}
The main question of this paper is whether dynamic compilation can be used to reduce potential side-channel information leakage?

Here, software diversity is explored as a defense against side-channel attacks
through dynamically randomizing the control flow of programs. Traditionally,
software diversification focuses on randomizing the program's representation,
such as the in-memory addresses of data. However, side-channel attacks rely on
dynamic properties of programs such as execution time and memory latencies.
This calls for a software diversification technique that randomizes the
program's execution as opposed to representation. 

Pre-existing diversification techniques may already have an impact on side-channels. For example, even re-ordering functions can have a large effect on
cache usage and performance. However, static compile time or load time
diversity is not enough since an attacker can simply use the static target
binary to generate a profile of the runtime characteristics of the program.
Even re-diversifying and switching to a new variant during execution is
insufficient as side-channels are fast enough to complete between
re-diversification cycles. Instead, the authors create a diversification model
which creates a version of the program consisting of replicated code fragments
with randomized control flow to switch between alternative replicas at runtime.
The program can now take numerous different paths which prevents the attacker
from developing a reliable model of program behavior. 

The randomized variants of a program fragment are made using diversifying 
transformations. This results in functionally equivalent program fragments that
have differing runtime characteristics. These program fragments are then
integrated into a program that dynamically chooses a control flow path at
runtime. 

\subsection{Case Study -- Mitigating Cache Side-Channel Attacks}

Pre-existing side-channel attacks against the AES encryption scheme exploit
cache behavior to determine the secret key. The authors implemented these
attacks against AES-128 routine in libcrypt 1.6.1. Two cache based side-channel
attacks were evaluated, one which was able to recover $96\%$ and the other $82\%$ of
the secret bits for a set number of iterations in the base case. Code
diversification was demonstrated to have an impact on the success rate of both
attacks. Static diversification performed decently, reducing the number of
correctly guessed bits to $104-108$ of $128$ for the first attack but to $52$ for the
second. However, the dynamic approach presented by the authors mitigated the
side-channels much more powerfully, reducing the number of correctly guessed
bits to $20$ for the first attack, and to $14$ for the second. 
\subsection{Conclusion}
This in itself demonstrates the potential for compiler-based code diversification to
protect against side-channel attacks, which could be a promising avenue for
greater security.
